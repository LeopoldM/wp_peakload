
\documentclass{class}

\usepackage{bbm}

\usetikzlibrary{positioning}


\begin{document}


\begingroup
\allowdisplaybreaks

%============================
% Setup and KKT conditions
%============================
\paragraph{Setup and notation.}
For each category $i$ with population weight $\mu_i$, type $\theta\in\Theta_i$, and state $s$, let $q_i(\theta,s)$ and $t_i(\theta,s)$ denote the allocation and transfer. Let $u(q,\theta,s)$ denote the \emph{marginal surplus} (marginal WTP net of marginal cost), with $u_q(q,\theta,s)<0$. The capacity constraint binds in on-peak states with multiplier $\lambda(s)$; the budget constraint has multiplier $\beta\ge 0$. For each $(i,\theta)$, let $\rho_i(\theta)\ge 0$ be the multiplier on IR and $\phi_i(\theta,s)\ge 0$ on the nonnegativity of transfers (if present). Let $N^{nb}$ be the set of categories with \emph{slack} IR (all types), and $N^b$ those with \emph{binding} IR (contributing categories). Integrals over types are w.r.t.\ $G_i$.

\paragraph{FOCs (stated as equalities to $0$).}
From the KKT conditions:
\begin{align}
\label{KKT-q}\tag{KKT-$q$}
(\mu_i\nu_i+\rho_i(\theta))\,u\!\left(q_i(\theta,s),\theta,s\right)\;-\;\mu_i\lambda(s) \;=\; 0,
\\
\label{KKT-t}\tag{KKT-$t$}
-\mu_i\nu_i + \mu_i\beta - \rho_i(\theta) + \phi_i(\theta,s) \;=\; 0.
\end{align}
If IR is slack for $i$, then $\rho_i(\theta)=0$ for all $\theta$.
If $t_i(\theta,s)>0$ then $\phi_i(\theta,s)=0$.

%============================
% Comparative statics in k
%============================
\paragraph{Comparative statics in $k$.}
Differentiate \eqref{KKT-q} w.r.t.\ $k$ and set to zero:
\[
(\mu_i\nu_i+\rho_i)\,u_q(q_i)\,\partial_k q_i \;+\; (\partial_k\rho_i)\,u(q_i,\theta,s) \;-\; \mu_i\,\partial_k\lambda(s) \;=\; 0,
\]
hence
\begin{equation}\label{dqdk-general}
\partial_k q_i(\theta,s)
=
\frac{\mu_i\,\partial_k\lambda(s)\;-\;(\partial_k\rho_i(\theta))\,u(q_i,\theta,s)}
{(\mu_i\nu_i+\rho_i(\theta))\,u_q(q_i,\theta,s)}.
\end{equation}

\emph{Slack IR} ($i\in N^{nb}$): $\rho_i=\partial_k\rho_i=0$, so
\begin{equation}\label{dqdk-slack}
\partial_k q_i \;=\; \frac{\partial_k\lambda(s)}{\nu_i\,u_q(q_i,\theta,s)}.
\end{equation}

\emph{Binding IR} ($i\in N^b$ with $t_i>0$): from \eqref{KKT-t} with $\phi_i=0$,
$\mu_i\beta=\mu_i\nu_i+\rho_i(\theta)\Rightarrow \rho_i(\theta)=\mu_i(\beta-\nu_i)$, so
$\partial_k\rho_i(\theta)=\mu_i\,\partial_k\beta$. Plugging into \eqref{dqdk-general}:
\begin{equation}\label{dqdk-binding}
\partial_k q_i \;=\; \frac{\partial_k\lambda(s)-\partial_k\beta\,u(q_i,\theta,s)}{\beta\,u_q(q_i,\theta,s)}.
\end{equation}

If the marginal category $i^\ast$ has slack IR and $t_{i^\ast}>0$, then from \eqref{KKT-t}
$\beta=\nu_{i^\ast}$; locally in $k$ this implies $\partial_k\beta=0$.

%============================
% Capacity balance and ∂k λ
%============================
\paragraph{Capacity balance and $\partial_k\lambda(s)$.}
Differentiating the capacity constraint gives
\[
1 \;=\; \sum_i \mu_i \int_{\Theta_i}\partial_k q_i(\theta,s)\,dG_i(\theta).
\]
Split the sum over $N^{nb}$ (use \eqref{dqdk-slack}) and $N^b$ (use \eqref{dqdk-binding}). Define
\begin{align}
\label{A-def}\tag{$\mathcal A$}
\mathcal A(s) \;&:=\;
\sum_{i\in N^{nb}}\mu_i\!\int_{\Theta_i}\frac{1}{\nu_i\,u_q(q_i,\theta,s)}\,dG_i
\;+\;
\sum_{i\in N^{b}}\mu_i\!\int_{\Theta_i}\frac{1}{\beta\,u_q(q_i,\theta,s)}\,dG_i,
\\
\label{B-def}\tag{$\mathcal B$}
\mathcal B(s) \;&:=\;
\sum_{i\in N^{b}}\mu_i\!\int_{\Theta_i}\frac{u(q_i,\theta,s)}{\beta\,u_q(q_i,\theta,s)}\,dG_i,
\\
\label{C-def}\tag{$\mathcal C$}
\mathcal C(s) \;&:=\;
\sum_{i\in N^{b}}\mu_i\!\int_{\Theta_i}\frac{u(q_i,\theta,s)^2}{\beta\,u_q(q_i,\theta,s)}\,dG_i.
\end{align}
Then
\begin{equation}\label{dlambdadk}
\partial_k\lambda(s) \;=\; \frac{1}{\mathcal A(s)} \;+\; \partial_k\beta\;\frac{\mathcal B(s)}{\mathcal A(s)}.
\end{equation}
Since $u_q<0$, we have $\mathcal A(s)<0$. If $\partial_k\beta=0$ (e.g., marginal category slack), then $\partial_k\lambda(s)=1/\mathcal A(s)<0$.

%============================
% Sign of ∂k q_i and a sufficient condition
%============================
\paragraph{Sign of $\partial_k q_i$ for contributing categories.}
For $i\in N^b$, combining \eqref{dqdk-binding} and \eqref{dlambdadk}:
\[
\partial_k q_i
=
\frac{1}{\beta u_q}\Bigg(
\frac{1}{\mathcal A}
\;+\;
\partial_k\beta\Big(\tfrac{\mathcal B}{\mathcal A}-u(q_i,\theta,s)\Big)
\Bigg).
\]
A sufficient condition for $\partial_k q_i>0$ is:
\[
\partial_k\beta<0
\quad\text{and}\quad
\tfrac{\mathcal B(s)}{\mathcal A(s)}-u(q_i,\theta,s)>0.
\]

%============================
% Determining ∂k β
%============================
\paragraph{Identifying $\partial_k\beta$.}
Aggregate IR and revenue. Using $t_i=0$ for $i\notin N^b$ and differentiating the revenue requirement,
\[
\sum_{i\in N^b}\mu_i\int_s\!\!\int_{\Theta_i} u(q_i,\theta,s)\,\partial_k q_i(\theta,s)\,dG_i\,dF(s) \;=\; I'(k).
\]
Substitute \eqref{dqdk-binding} and rearrange using \eqref{dlambdadk} and the definitions \eqref{A-def}–\eqref{C-def} to obtain
\begin{equation}\label{dbetadk}
\partial_k\beta
\;=\;
\frac{ I'(k)\;-\;\displaystyle\int_s \frac{\mathcal B(s)}{\mathcal A(s)}\,dF(s) }
{ \displaystyle\int_s \Big( \frac{\mathcal B(s)^2}{\mathcal A(s)} - \mathcal C(s) \Big)\,dF(s) }.
\end{equation}

\paragraph{Denominator positivity.}
Write $r(q,\theta,s):=-u_q(q,\theta,s)>0$. Over $N^b$,
\[
\mathcal A_b(s)=-\frac{1}{\beta}\sum_{i\in N^b}\mu_i\!\int\!\frac{1}{r}\,,\quad
\mathcal B(s)=-\frac{1}{\beta}\sum_{i\in N^b}\mu_i\!\int\!\frac{u}{r}\,,\quad
\mathcal C(s)=-\frac{1}{\beta}\sum_{i\in N^b}\mu_i\!\int\!\frac{u^2}{r}\,.
\]
Cauchy–Schwarz yields $\big(\sum\!\int \tfrac{u}{r}\big)^2 \le \big(\sum\!\int \tfrac{1}{r}\big)\big(\sum\!\int \tfrac{u^2}{r}\big)$,
hence $\mathcal B(s)^2 \le \mathcal A_b(s)\,\mathcal C(s)$.
Since $\mathcal A(s)\le \mathcal A_b(s)<0$,
\[
\frac{\mathcal B(s)^2}{\mathcal A(s)} \;\ge\; \frac{\mathcal B(s)^2}{\mathcal A_b(s)} \;\ge\; \mathcal C(s),
\]
so the denominator of \eqref{dbetadk} is strictly positive.

%============================
% Linear-in-q specialization with normalized averages
%============================
\paragraph{Linear-in-$q$ specialization with normalized averages.}
Assume $u(q,\theta,s)=a(\theta,s)+u_q(s)\,q$ with $u_q(s)<0$ and $u_{q\theta}=u_{qs}=0$. Let
\[
M^{nb}:=\sum_{i\in N^{nb}}\mu_i\int_{\Theta_i}\!dG_i(\theta),
\qquad
M^{b}:=\sum_{i\in N^{b}}\mu_i\int_{\Theta_i}\!dG_i(\theta),
\]
and define
\[
w^{nb}:=\sum_{i\in N^{nb}}\frac{\mu_i}{\nu_i}\int_{\Theta_i}\!dG_i(\theta),
\qquad
w^{b}:=\frac{M^{b}}{\beta},
\qquad
w:=w^{nb}+w^{b}.
\]
Introduce the \emph{normalized} average over contributing types:
\[
\overline{u}^{\,b}(s)\;:=\;\frac{1}{M^{b}}\sum_{i\in N^{b}}\mu_i\int_{\Theta_i} u(q_i(\theta,s),\theta,s)\,dG_i(\theta),
\quad
\overline{u^{2}}{}^{\,b}(s)\;:=\;\frac{1}{M^{b}}\sum_{i\in N^{b}}\mu_i\int_{\Theta_i} u(\cdot)^2\,dG_i.
\]
Then the objects in \eqref{A-def}–\eqref{C-def} simplify to
\begin{equation}\label{ABC-linear}
\mathcal A(s)=\frac{w}{u_q(s)},\qquad
\mathcal B(s)=\frac{w^{b}}{u_q(s)}\,\overline{u}^{\,b}(s),\qquad
\mathcal C(s)=\frac{w^{b}}{u_q(s)}\,\overline{u^{2}}{}^{\,b}(s),
\end{equation}
and therefore
\begin{equation}\label{dlambda-linear}
\partial_k\lambda(s)=\frac{u_q(s)}{w}\;+\;\frac{w^{b}}{w}\,\partial_k\beta\;\overline{u}^{\,b}(s).
\end{equation}
(Notice the use of \emph{normalized} averages: this removes unintended scale factors and makes identities like $\mathcal B/\mathcal A=(w^{b}/w)\,\overline{u}^{\,b}$ exact.)

\paragraph{Clean derivatives of $\mathcal A$ and $\mathcal B$.}
Since $u_q(s)$ is $k$-independent here,
\[
\partial_k \mathcal A(s)=\frac{\partial_\beta w}{u_q(s)}\,\partial_k\beta
\quad\text{with}\quad
\partial_\beta w=\partial_\beta w^{b}=-\frac{M^{b}}{\beta^2}<0.
\]
For $\mathcal B(s)=(w^{b}/u_q)\,\overline{u}^{\,b}(s)$,
\[
\partial_k \mathcal B(s)
=
\frac{\partial_\beta w^{b}}{u_q(s)}\,\overline{u}^{\,b}(s)\,\partial_k\beta
\;+\;
\frac{w^{b}}{u_q(s)}\,\partial_k \overline{u}^{\,b}(s).
\]
Under linear $u$, $\partial_k \overline{u}^{\,b}(s)=u_q(s)\,\overline{\partial_k q}^{\,b}(s)$, where
\[
\overline{\partial_k q}^{\,b}(s):=
\frac{1}{M^{b}}\sum_{i\in N^{b}}\mu_i\int_{\Theta_i}\partial_k q_i(\theta,s)\,dG_i(\theta).
\]
Thus
\begin{equation}\label{dBdk-clean}
\partial_k \mathcal B(s)
=
\frac{\partial_\beta w^{b}}{u_q(s)}\,\overline{u}^{\,b}(s)\,\partial_k\beta
\;+\;
w^{b}\,\overline{\partial_k q}^{\,b}(s).
\end{equation}
Equations \eqref{dlambda-linear}–\eqref{dBdk-clean} give a compact, scale-free decomposition of the response of $(\lambda,\mathcal A,\mathcal B)$ in terms of $\partial_k\beta$ and the average quantity response among contributors.

\endgroup

% weights
w_i^{nb}(\theta,s):=\frac{1}{\nu_i\,u_q(q_i(\theta,s),\theta,s)},\quad
w_i^{b}(\theta,s):=\frac{1}{\beta\,u_q(q_i(\theta,s),\theta,s)}.

% total weights
\tag{$Z(s)$}\label{Z-def}
Z(s):=\sum_{i\in N^{nb}}\mu_i\!\int_{\Theta_i} w_i^{nb}(\theta,s)\,dG_i(\theta)
     +\sum_{i\in N^{b}}\mu_i\!\int_{\Theta_i} w_i^{b}(\theta,s)\,dG_i(\theta),
\]
\[
\tag{$Z_b(s)$}\label{Zb-def}
Z_b(s):=\sum_{i\in N^{b}}\mu_i\!\int_{\Theta_i} w_i^{b}(\theta,s)\,dG_i(\theta).
\]

% expectations over the binding set
\tag{$\mathbb{E}_{w}[u\,|\,s]$}\label{Ew-def}
\mathbb{E}_{w}[u\,|\,s]
:= \frac{1}{Z_b(s)}
\sum_{i\in N^{b}}\mu_i\!\int_{\Theta_i} w_i^{b}(\theta,s)\,
u\!\big(q_i(\theta,s),\theta,s\big)\,dG_i(\theta),
\]
\[
\tag{$\mathbb{E}_{w}[u^2\,|\,s]$}\label{Ew2-def}
\mathbb{E}_{w}[u^2\,|\,s]
:= \frac{1}{Z_b(s)}
\sum_{i\in N^{b}}\mu_i\!\int_{\Theta_i} w_i^{b}(\theta,s)\,
u\!\big(q_i(\theta,s),\theta,s\big)^2\,dG_i(\theta).




\section{Proof of Corollary XXXX (on-peak / off-peak version)}

\begin{proof}
For each \((\theta,i,s)\), the KKT conditions are
\begin{align}
\label{KKT-q}\tag{KKT-$q$}
(\mu_i\nu_i+\rho_i)\,u\!\left(q_i(\theta,s),\theta,s\right)-\mu_i\lambda(s)=0,
\\
\label{KKT-t}\tag{KKT-$t$}
-\mu_i\nu_i+\mu_i\beta-\rho_i+\phi_i(\theta,s)=0,
\end{align}
with \(\rho_i(\theta)\ge 0\) (IR), \(\phi_i(\theta,s)\ge 0\) (\(t_i\ge 0\)), and \(\phi_i(\theta,s)=0\) whenever \(t_i(\theta,s)>0\).

Let \(S^\ast\) be the set of \emph{off-peak} states (capacity slack) and \(T^\ast\) the set of \emph{on-peak} states (capacity binding), i.e.
\[
s\in S^\ast \iff \lambda(s)=0,
\qquad
s\in T^\ast \iff \lambda(s)>0.
\]

\paragraph{Step 1: Individual derivatives.}
Differentiating \eqref{KKT-q} w.r.t.\ \(k\) (holding \(s\) fixed) gives
\[
(\mu_i\nu_i+\rho_i)\,u_q(q_i)\,\partial_k q_i(\theta,s)
\;+\;(\partial_k\rho_i)\,u(q_i,\theta,s)
\;-\;\mu_i\,\partial_k\lambda(s)=0,
\]
so
\begin{equation}\label{dqdk-general}
\partial_k q_i(\theta,s)
=
\frac{\mu_i\,\partial_k\lambda(s)\;-\;(\partial_k\rho_i)\,u(q_i,\theta,s)}
{(\mu_i\nu_i+\rho_i)\,u_q(q_i,\theta,s)}.
\end{equation}

If \(i\) has \emph{slack IR} (\(\rho_i=0\) for all \(\theta\)), then \(\partial_k\rho_i=0\) and
\begin{equation}\label{dqdk-slack}
\partial_k q_i(\theta,s)=\frac{\partial_k\lambda(s)}{\nu_i\,u_q(q_i,\theta,s)}.
\end{equation}

If \(i\) has \emph{binding IR} and \(t_i>0\) (so \(\phi_i=0\)), then from \eqref{KKT-t}:
\(\mu_i\beta=\mu_i\nu_i+\rho_i\), i.e.\ \(\rho_i=\mu_i(\beta-\nu_i)\) and \(\partial_k\rho_i=\mu_i\,\partial_k\beta\).
Plugging into \eqref{dqdk-general} yields
\begin{equation}\label{dqdk-binding}
\partial_k q_i(\theta,s)
=\frac{\partial_k\lambda(s)\;-\;\partial_k\beta\,u(q_i,\theta,s)}
{\beta\,u_q(q_i,\theta,s)}.
\end{equation}

\paragraph{Step 2: State-by-state capacity (on vs off peak).}
For each \(s\), the capacity constraint is \(\sum_i \mu_i \mathbb{E}_\theta[q_i(\theta,s)]\le k\).
Hence
\[
s\in T^\ast:\quad \sum_i \mu_i \mathbb{E}_\theta[\partial_k q_i(\theta,s)] = 1,
\qquad
s\in S^\ast:\quad \sum_i \mu_i \mathbb{E}_\theta[\partial_k q_i(\theta,s)] = 0.
\]

\paragraph{Step 3: Solving for \(\partial_k\lambda(s)\) when \(s\in T^\ast\).}
Split categories into \(N^{nb}\) (IR slack) and \(N^{b}\) (IR binding with \(t_i>0\)).
Define, for each \((\theta,s)\),
\[
w^{nb}_i(\theta,s)
:=\frac{1}{\nu_i\,u_q(q_i(\theta,s),\theta,s)},
\qquad
w^{b}_i(\theta,s)
:=\frac{1}{\beta\,u_q(q_i(\theta,s),\theta,s)}.
\]
Aggregate, for each state \(s\),
\[
\bar{w}^{nb}(s):=\sum_{i\in N^{nb}}\mu_i\,\mathbb{E}_\theta\!\left[w^{nb}_i(\theta,s)\right],
\quad
\bar{w}^{b}(s):=\sum_{i\in N^{b}}\mu_i\,\mathbb{E}_\theta\!\left[w^{b}_i(\theta,s)\right],
\quad
\bar{w}(s):=\bar{w}^{nb}(s)+\bar{w}^{b}(s).
\]
Over \(N^{b}\), define the weighted moments
\begin{align}
\label{Eu-def}\tag{$\mathbb{E}_w[u\,|\,s]$}
\mathbb{E}_{w}[u\,|\,s]
&:=\frac{1}{\bar{w}^{b}(s)}
\sum_{i\in N^{b}}\mu_i\,\mathbb{E}_\theta\!\Big[w^{b}_i(\theta,s)\,u(q_i,\theta,s)\Big],\\
\label{Eu2-def}\tag{$\mathbb{E}_w[u^2\,|\,s]$}
\mathbb{E}_{w}[u^2\,|\,s]
&:=\frac{1}{\bar{w}^{b}(s)}
\sum_{i\in N^{b}}\mu_i\,\mathbb{E}_\theta\!\Big[w^{b}_i(\theta,s)\,u(q_i,\theta,s)^2\Big].
\end{align}
Plugging \eqref{dqdk-slack} and \eqref{dqdk-binding} into
\(\sum_i \mu_i \mathbb{E}_\theta[\partial_k q_i(\theta,s)] = 1\) for \(s\in T^\ast\) gives
\begin{equation}\label{dlambdadk}
\partial_k\lambda(s)
\;=\;
\frac{1}{\bar{w}(s)}
\;+\;
\partial_k\beta\;\frac{\bar{w}^{b}(s)}{\bar{w}(s)}\,\mathbb{E}_{w}[u\,|\,s],
\qquad s\in T^\ast,
\end{equation}
and since \(u_q<0\), we have \(\bar{w}(s)<0\).

\paragraph{Step 4: Final derivatives on-peak vs off-peak.}
Combining \eqref{dqdk-binding}–\eqref{dlambdadk} for \(s\in T^\ast\),
\[
\partial_k q_i(\theta,s)
=
\frac{1}{\beta\,u_q(q_i,\theta,s)}
\left(
\frac{1}{\bar{w}(s)}
+
\partial_k\beta\Big[
\frac{\bar{w}^{b}(s)}{\bar{w}(s)}\,\mathbb{E}_{w}[u\,|\,s]
-
u(q_i,\theta,s)
\Big]
\right),
\qquad s\in T^\ast.
\]
For \(s\in S^\ast\) (off-peak), \(\lambda(s)=0\) and \eqref{KKT-q} implies \(u(q_i(\theta,s),\theta,s)=0\).
Hence
\[
\partial_k\lambda(s)=0
\quad\text{and}\quad
\partial_k q_i(\theta,s)=0,
\qquad s\in S^\ast.
\]

\paragraph{Step 5: Marginal category \(i^\ast\) with slack IR.}
If the marginal category \(i^\ast\) has slack IR and \(t_{i^\ast}>0\) on an interval of \(k\) where \(i^\ast\) remains marginal, then \eqref{KKT-t} implies \(\beta=\nu_{i^\ast}\), so \(\partial_k\beta=0\) on that interval. For \(s\in T^\ast\),
\[
\partial_k q_i(\theta,s)
=\frac{1}{\beta\,u_q(q_i,\theta,s)}\cdot\frac{1}{\bar{w}(s)},
\qquad
\text{and }\ \partial_k q_i(\theta,s)=0 \ \text{for } s\in S^\ast.
\]
Because \(u_q<0\) and \(\bar{w}(s)<0\), we have \(\partial_k q_i(\theta,s)>0\) on-peak and zero off-peak.

This completes the on-peak/off-peak adaptation: only peak states \(s\in T^\ast\) respond to \(k\); off-peak states \(s\in S^\ast\) show no marginal response in \(\lambda\) or allocations.
\end{proof}


% --- Up to and including the proof of \partial_k \beta (on-peak / off-peak) ---

\paragraph{FOCs.}
\begin{align}
\label{KKT-q}\tag{KKT-$q$}
(\mu_i\nu_i+\rho_i)\,u\!\left(q_i(\theta,s),\theta,s\right)-\mu_i\lambda(s)=0,
\\
\label{KKT-t}\tag{KKT-$t$}
-\mu_i\nu_i+\mu_i\beta-\rho_i+\phi_i(\theta,s)=0.
\end{align}
If IR is slack for $i$, then $\rho_i=0$ for all $\theta$. If $t_i(\theta,s)>0$ then $\phi_i(\theta,s)=0$.

Let $S^\ast$ denote \emph{off-peak} states (capacity slack) and $T^\ast$ \emph{on-peak} states (capacity binding):
\[
s\in S^\ast \iff \lambda(s)=0,\qquad s\in T^\ast \iff \lambda(s)>0.
\]

\paragraph{Step 1: Individual derivatives.}
Differentiate \eqref{KKT-q} w.r.t.\ $k$:
\[
(\mu_i\nu_i+\rho_i)\,u_q(q_i)\,\partial_k q_i(\theta,s)
\;+\; (\partial_k\rho_i)\,u(q_i,\theta,s)
\;-\; \mu_i\,\partial_k\lambda(s) \;=\; 0,
\]
hence
\begin{equation}\label{dqdk-general}
\partial_k q_i(\theta,s)
=
\frac{\mu_i\,\partial_k\lambda(s)\;-\;(\partial_k\rho_i)\,u(q_i,\theta,s)}
{(\mu_i\nu_i+\rho_i)\,u_q(q_i,\theta,s)}.
\end{equation}

For $i\in N^{nb}$ (IR slack), $\rho_i=\partial_k\rho_i=0$, so
\begin{equation}\label{dqdk-slack}
\partial_k q_i(\theta,s)
\;=\; \frac{\partial_k\lambda(s)}{\nu_i\,u_q(q_i,\theta,s)}.
\end{equation}

For $i\in N^{b}$ (IR binding) with $t_i>0$, from \eqref{KKT-t} and $\phi_i=0$:
$\mu_i\beta=\mu_i\nu_i+\rho_i \Rightarrow \rho_i=\mu_i(\beta-\nu_i)$ and thus $\partial_k\rho_i=\mu_i\,\partial_k\beta$.
Plugging into \eqref{dqdk-general}:
\begin{equation}\label{dqdk-binding}
\partial_k q_i(\theta,s)
\;=\;
\frac{\partial_k\lambda(s)-\partial_k\beta\,u(q_i,\theta,s)}{\beta\,u_q(q_i,\theta,s)}.
\end{equation}

\paragraph{Step 2: Capacity by state (on vs off peak).}
For each state $s$,
\[
\sum_i \mu_i \mathbb{E}_\theta[q_i(\theta,s)] \le k.
\]
Therefore,
\[
s\in T^\ast:\quad \sum_i \mu_i \mathbb{E}_\theta[\partial_k q_i(\theta,s)] = 1,
\qquad
s\in S^\ast:\quad \sum_i \mu_i \mathbb{E}_\theta[\partial_k q_i(\theta,s)] = 0.
\]
(In off-peak states the constraint is slack, so a marginal change in $k$ does not move the optimum; at the optimum, \eqref{KKT-q} also implies $u(q_i(\theta,s),\theta,s)=0$ and thus $\partial_k q_i(\theta,s)=\partial_k\lambda(s)=0$ for $s\in S^\ast$.)

\paragraph{Step 3: Solving $\partial_k\lambda(s)$ on-peak.}
Define for every $(\theta,s)$
\[
w^{nb}_i(\theta,s) := \frac{1}{\nu_i\,u_q(q_i(\theta,s),\theta,s)},
\qquad
w^{b}_i(\theta,s) := \frac{1}{\beta\,u_q(q_i(\theta,s),\theta,s)}.
\]
Aggregate by state $s$:
\[
\bar{w}^{nb}(s) := \sum_{i\in N^{nb}}\mu_i\,\mathbb{E}_\theta\!\big[w^{nb}_i(\theta,s)\big],
\qquad
\bar{w}^{b}(s) := \sum_{i\in N^{b}}\mu_i\,\mathbb{E}_\theta\!\big[w^{b}_i(\theta,s)\big],
\qquad
\bar{w}(s)=\bar{w}^{nb}(s)+\bar{w}^{b}(s).
\]
Weighted moments (over $N^{b}$) for each on-peak state $s$:
\begin{align}
\label{Eu-def}\tag{$\mathbb{E}_w[u\,|\,s]$}
\mathbb{E}_{w}[u\,|\,s]
&:=\frac{1}{\bar{w}^{b}(s)}
\sum_{i\in N^{b}}\mu_i\,\mathbb{E}_\theta\!\Big[w^{b}_i(\theta,s)\,u(q_i,\theta,s)\Big],\\
\label{Eu2-def}\tag{$\mathbb{E}_w[u^2\,|\,s]$}
\mathbb{E}_{w}[u^2\,|\,s]
&:=\frac{1}{\bar{w}^{b}(s)}
\sum_{i\in N^{b}}\mu_i\,\mathbb{E}_\theta\!\Big[w^{b}_i(\theta,s)\,u(q_i,\theta,s)^2\Big].
\end{align}
Plugging \eqref{dqdk-slack} and \eqref{dqdk-binding} into
$\sum_i \mu_i \mathbb{E}_\theta[\partial_k q_i(\theta,s)] = 1$ for $s\in T^\ast$ gives
\begin{equation}\label{dlambdadk}
\partial_k\lambda(s)
\;=\;
\frac{1}{\bar{w}(s)} \;+\; \partial_k\beta\;\frac{\bar{w}^{b}(s)}{\bar{w}(s)}\,\mathbb{E}_{w}[u\,|\,s],
\qquad s\in T^\ast,
\end{equation}
with $\bar{w}(s)<0$ since $u_q<0$.

\paragraph{Step 4: Up to the derivative of $\beta$.}
If the marginal category $i^\ast$ has slack IR and remains marginal as $k$ varies (with $t_{i^\ast}>0$), then from \eqref{KKT-t} we have $\beta=\nu_{i^\ast}$ and hence $\partial_k\beta=0$ on that interval.

To derive $\partial_k\beta$ in general, use that transfers are zero for $i\notin N^{b}$ and differentiate the budget constraint:
\[
\sum_{i\in N^{b}} \mu_i\,\mathbb{E}_{s\in T^\ast}\mathbb{E}_\theta\!\left[\,u(q_i,\theta,s)\,\partial_k q_i(\theta,s)\,\right]
\;=\; I'(k),
\]
where off-peak states contribute zero (both $u=0$ and $\partial_k q_i=0$). Substitute
\eqref{dqdk-binding} and eliminate $\partial_k\lambda(s)$ using \eqref{dlambdadk} and the
definitions \eqref{Eu-def}–\eqref{Eu2-def}. After rearrangement,
\begin{equation}\label{dbetadk}
\partial_k\beta
\;=\;
\frac{\, I'(k)\;-\;\displaystyle \mathbb{E}_{s\in T^\ast}\!\left[\frac{\bar{w}^{b}(s)}{\bar{w}(s)}\,\mathbb{E}_{w}[u\,|\,s]\right]\,}
{\,\displaystyle \mathbb{E}_{s\in T^\ast}\!\left[
\frac{\big(\bar{w}^{b}(s)\,\mathbb{E}_{w}[u\,|\,s]\big)^{2}}{\bar{w}(s)}
\;-\;
\bar{w}^{b}(s)\,\mathbb{E}_{w}[u^{2}\,|\,s]
\right]\,}.
\end{equation}
% (End: up to and including the proof of \partial_k \beta.)

\end{document}