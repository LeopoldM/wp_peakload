\documentclass{class}

\begin{document}

\section{Priority service and contingent nonlinear pricing: an economics-focused genealogy}\label{app:genealogy_priority_contingent}

\subsection{Seed papers: one-paragraph deep contributions}\label{app:seed_papers}

\paragraph{\textbf{Wilson (1989, Econometrica).}}
Wilson formalizes rationing of scarce, nonstorable capacity as an equilibrium allocation problem in which ``priority'' (delivery order / reliability) is an endogenous contract attribute, and shows how incentive-compatible menus of priority/quality levels can decentralize efficient rationing when implementing state-contingent spot pricing is infeasible or costly. The deep contribution is to treat reliability differentiation as a market form (product-line choice under congestion) rather than an ad hoc rule, derive the pricing/assignment logic that supports efficiency and competition with rationing, and thereby provide a bridge between public pricing problems (how to ration efficiently) and competitive provision (when and how priority differentiation survives under noncooperative supply).

\paragraph{\textbf{Spulber (1992, International Economic Review).}}
Spulber extends nonlinear pricing to environments with demand/cost/capacity uncertainty by modeling rationing rules as complete contingent contracts: consumers select a single ex ante reference point (e.g., baseload/firm demand) that pins down both payment and the ex post, state-contingent allocation once uncertainty is realized. The deep contribution is a mechanism-design implementation argument: the reference-point choice elicits private information and implements the Pareto-optimal state-contingent allocation within a nonlinear-tariff framework, and the paper then derives practical near-efficient approximations (discrete customer classes; linear prorating) that sharply reduce implementation/transaction costs relative to fully contingent nonlinear tariffs.

\paragraph{\textbf{Spulber (1992, Journal of Regulatory Economics).}}
Spulber brings the contingent-contract logic into second-best regulated pricing (break-even / revenue adequacy with demand and capacity shocks) and characterizes the second-best Pareto-optimal mechanism as capacity-contingent nonlinear pricing, with equivalent implementations via (i) rationing based on a consumer-chosen minimum/firm level and (ii) generalized priority pricing. The deep contribution is to unify regulated nonlinear pricing with reliability/priority mechanisms, show how ``firm power'' self-selection implements the efficient state-contingent rationing rule under regulatory constraints, and provide approximation/welfare guidance (e.g., small menus of classes capturing most gains) that connects the theory tightly to interruptible/curtailable service design.

\subsection{Descendant literature (economics): core branches and deep contributions}\label{app:descendants}

\subsubsection{A. Regulation and utility pricing: self-rationing, reliability differentiation, and demand response}\label{app:branch_A}

\paragraph{\textbf{Wilson (1989, JRE).}}
Extends the priority-service idea into a Ramsey/second-best pricing environment, deriving how priority/reliability differentiation and associated markups should be structured when the provider faces a revenue requirement. The deep contribution is to integrate classic Ramsey logic with congestion-based quality differentiation, clarifying the wedge between marginal-cost pricing and implementable priority menus under regulatory constraints.

\paragraph{\textbf{Woo (1990, JRE).}}
Develops an electricity-pricing framework in which customers can self-ration (voluntarily accept interruption/curtailment) to manage capacity scarcity and uncertainty, and characterizes when such self-rationing tariffs improve welfare relative to uniform service. The deep contribution is to translate priority/firm-service ideas into an electricity setting with explicit rationing technology and to show how tariff design can induce efficient sorting between firm and interruptible service when capacity is scarce.

\paragraph{\textbf{Strauss \& Oren (1993, The Energy Journal).}}
Designs a priority-pricing tariff for interruptible electric service that lets customers self-select both interruption priority and an ``early notification'' option, thereby matching interruption costs to rationing order. The deep contribution is to enrich the Wilson/Spulber priority mechanism with an additional service attribute (notification lead time) that matters for real interruption damages, and to show how a small contract menu can implement an economically meaningful reliability differentiation.

\paragraph{\textbf{Doucet, Jo, Min, Roland \& Strauss (1996, Energy Economics).}}
Studies electricity rationing implemented through a two-stage mechanism (ex ante contracting and ex post rationing) and characterizes how staged commitments can improve allocative performance under scarcity relative to purely ex post rules. The deep contribution is to formalize staged rationing as a tractable alternative to full state-contingent pricing, preserving much of the contingent-contract logic while respecting implementation constraints.

\paragraph{\textbf{Seeto, Woo \& Horowitz (1997, Energy Economics).}}
Analyzes customer choice between time-of-use pricing and Hopkinson-type demand charges, emphasizing how tariff form affects load shifting and peak reduction in regulated distribution settings. The deep contribution is to connect contingent/self-rationing intuitions to empirically relevant tariff instruments (TOU versus demand-charge alternatives), clarifying which dimensions of demand variability and customer heterogeneity drive the tariff-selection and welfare ranking.

\paragraph{\textbf{Woo, Horowitz \& Martin (1998, JRE).}}
Introduces reliability differentiation for electricity transmission and studies how service classes with different reliability levels can be priced/allocated to ration constrained transmission capacity. The deep contribution is to transplant priority-service theory from generic congestible services into the network-reliability context of transmission, making reliability itself a priced attribute and clarifying the role of differentiated access to scarce network capacity.

\paragraph{\textbf{Bernard \& Roland (2000, Resource and Energy Economics).}}
Analyzes load-management programs offered alongside regular service, focusing on the efficiency consequences of cross-subsidies and transaction/participation costs when customers can opt into self-rationing programs. The deep contribution is to relax the ``only service available'' simplification: when firm and interruptible options coexist, the welfare performance of self-rationing depends critically on tariff interactions, fixed costs of participation, and cross-subsidization across customer groups.

\paragraph{\textbf{Woo, Horii \& Horowitz (2002, Managerial and Decision Economics).}}
Develops and evaluates the Hopkinson tariff as a practical alternative to TOU pricing, emphasizing its informational/implementation advantages and the way it targets peak demand through demand charges. The deep contribution is to interpret a widely used tariff instrument as a structured approximation to more complex contingent tariffs, linking feasibility constraints (metering, billing complexity) to the design space implied by the contingent-contract view.

\paragraph{\textbf{Horowitz \& Woo (2006, Energy).}}
Studies demand-response rate options and provides conditions under which new tariff menus can generate Pareto improvements by inducing efficient participation and load reduction while satisfying utility/revenue constraints. The deep contribution is to frame demand-response options as a mechanism-design problem with participation constraints, making explicit the policy-relevant tradeoffs between incentives, revenue adequacy, and welfare.

\paragraph{\textbf{Woo, Kollman, Orans, Price \& Horii (2008, Energy Policy).}}
Uses the rollout of advanced metering infrastructure (AMI) to discuss which new tariff menus and service options become implementable, emphasizing practical pathways beyond uniform pricing toward differentiated products and demand response. The deep contribution is to connect the contingent-contract/priorities logic to technology constraints: AMI expands the feasible set of contract instruments that approximate state-contingent allocation.

\paragraph{\textbf{Greening (2010, Energy).}}
Focuses on institutional design: who should implement demand-response resources in deregulated markets, and how responsibilities/incentives affect realized efficiency gains. The deep contribution is to complement pricing mechanisms with governance: even if priority/contingent tariffs exist in theory, market structure and assignment of roles determine whether the intended rationing/response is delivered.

\paragraph{\textbf{Moore, Woo, Horii, Price \& Olson (2010, Energy).}}
Estimates the option value of a non-firm (interruptible) electricity tariff used by an LDC, interpreting interruptibility as an embedded option that allows curtailment during high wholesale-price hours. The deep contribution is to translate ``interruptible service = contingent contract'' into an empirically implementable valuation object, providing a welfare-relevant metric for how much flexibility is worth to the intermediary.

\paragraph{\textbf{Matsukawa (2006, MPRA).}}
Studies regulation of a monopoly offering priority service, comparing minimum reliability standards, price caps, and rate-of-return regulation and their effects on capacity investment and market penetration of differentiated reliability. The deep contribution is to integrate the priority-service product line with regulatory instruments, highlighting how regulation reshapes both the contract menu and endogenous capacity.

\paragraph{\textbf{Matsukawa (2009, JRE).}}
Extends the analysis of regulated priority service to quantify how different regulatory regimes affect reliability differentiation outcomes (capacity and penetration) relative to laissez-faire. The deep contribution is to provide a clean comparative-statics mapping from regulatory constraint to the equilibrium reliability menu and investment, operationalizing Wilson/Spulber ideas inside standard regulatory toolkits.

\paragraph{\textbf{Schroyen \& Oyenuga (2011, JRE).}}
Analyzes optimal pricing and capacity choice for a public service under risk of interruption, jointly choosing tariffs and capacity when reliability is stochastic. The deep contribution is to treat interruption risk as a primitive constraint that interacts with both pricing and investment, thereby endogenizing the reliability/capacity dimension central to priority service.

\paragraph{\textbf{Chao (2011, JRE).}}
Studies demand response in wholesale electricity markets through the lens of customer baselines, comparing administrative and contractual baseline designs and emphasizing gaming and efficiency implications. The deep contribution is to show that implementability hinges on contract design details: baselines function like a reference point in a contingent contract, and poor baseline design can create illusory reductions and distort rationing.

\paragraph{\textbf{Chao (2012, JRE).}}
Develops a theory of consumer subscription service that unifies priority service (ex ante hedging via service options) and dynamic pricing (ex post spot procurement) within a two-settlement structure. The deep contribution is to reinterpret priority service as a risk-management layer compatible with competitive wholesale markets, turning the ex ante reference-point idea into a market-design architecture.

\paragraph{\textbf{Crampes \& Léautier (2015, JRE).}}
Studies consumer participation in electricity adjustment (balancing) markets when consumers have private information about their value of electricity, deriving welfare/efficiency implications of opening these markets to demand-side resources. The deep contribution is to embed demand response in a post-contracting, shock-response setting and to identify the informational constraints that limit the first-best ``open the market'' intuition.

\paragraph{\textbf{Clastres \& Khalfallah (2015, Energy Economics).}}
Builds an analytical equilibrium model to characterize when activating demand elasticity via demand-response mechanisms raises welfare, emphasizing that welfare gains are confined to an ``optimal area'' of price signals. The deep contribution is to deliver sharp conditions for when DR is beneficial, bridging the mechanism view (prices as signals for rationing/response) with equilibrium feedbacks.

\paragraph{\textbf{Woo, Sreedharan, Hargreaves, Kahrl, Wang \& Horowitz (2014, Applied Energy).}}
Reviews electricity product differentiation, including reliability differentiation and tariff-based differentiation, synthesizing how utilities and markets segment service along multiple attributes. The deep contribution is to codify the ``priority/contingent pricing'' line as one pillar of a broader product-differentiation agenda in electricity, clarifying where theory maps into observed tariff and service innovations.

\paragraph{\textbf{Woo, Milstein, Tishler \& Zarnikau (2019, Energy Policy).}}
Proposes an electricity market design intended to address practical concerns (including missing money and manipulation), with a focus on implementable structures that preserve efficiency and stakeholder acceptability. The deep contribution is to emphasize mechanism feasibility under real market frictions and to position reliability/contract design as part of market architecture rather than only retail regulation.

\subsubsection{B. Capacity rights and advance-purchase/spot hybrids in infrastructure markets}\label{app:branch_B}

\paragraph{\textbf{David, Le Breton \& Mérillon (2007, IDEI WP).}}
Models a monopolist operating with advance-purchase and spot markets (motivated by natural gas transportation), and derives optimal pricing policies across the two stages under capacity and demand uncertainty. The deep contribution is to generalize the contingent-contract intuition to two-market (forward/spot) environments where advance commitments act as reference points that discipline ex post rationing and pricing.

\paragraph{\textbf{David, Le Breton \& Mérillon (2007, IDEI WP).}}
Studies public utility pricing and capacity choice with stochastic demand, jointly characterizing tariff structure and investment when demand uncertainty makes reliability/capacity constraints salient. The deep contribution is to treat capacity choice as part of the contract problem: tariffs must both allocate scarce capacity ex post and fund/incentivize capacity ex ante.

\subsubsection{C. IO spillovers: menus under individual demand uncertainty (advance purchase, flat rates, optional plans)}\label{app:branch_C}

\paragraph{\textbf{Gale \& Holmes (1993, AER).}}
Studies advance-purchase discounts as a capacity-allocation device under demand uncertainty, showing how intertemporal price discrimination sorts consumers and manages limited capacity. The deep contribution is to show that ``contingent allocation via early commitment'' is a general IO screening tool: advance purchase replicates part of the state-contingent contract logic without full contingency.

\paragraph{\textbf{Dana (1998, JPE).}}
Analyzes advance-purchase discounts and price discrimination in competitive markets with demand uncertainty, clarifying when intertemporal menus persist under competition and how they affect efficiency and surplus. The deep contribution is to position contingent/priority-style contracting as compatible with competition: even without market power, capacity/demand uncertainty can sustain menu-based sorting.

\paragraph{\textbf{Dana (2001, IER).}}
Studies monopoly price dispersion under demand uncertainty, providing a theory in which the seller offers multiple prices (or mechanisms) to manage uncertain demand realizations. The deep contribution is to connect demand uncertainty to endogenous pricing menus, supplying an IO analogue of contingent-contract rationing where dispersion substitutes for complete state contingency.

\paragraph{\textbf{Miravete (2002, ReStud).}}
Estimates demand for local telephone service with asymmetric information and optional calling plans, using the menu of plans to identify heterogeneity and information frictions. The deep contribution is empirical mechanism-design discipline: optional plans are interpreted as a screening device that resembles reference-point contingent contracting, and the paper demonstrates how to structurally recover preferences/information from observed tariff choices.

\paragraph{\textbf{Herweg \& Mierendorff (2013, JEEA).}}
Studies flat-rate tariffs under uncertain demand (and behavioral frictions such as loss aversion), explaining why simple tariffs can be privately optimal and persist despite apparent inefficiency. The deep contribution is to rationalize coarse approximations (flat rates) as optimal responses to uncertainty and behavioral constraints, providing an IO foundation for why ``simple'' tariff classes can be near-optimal in the Spulber sense.

\paragraph{\textbf{Reitman (1991, JIE).}}
Analyzes endogenous quality differentiation in congested markets, showing how congestion creates incentives to differentiate service quality and prices in equilibrium. The deep contribution is to provide an IO theory of congestion-induced product lines that complements Wilson: priority/quality differentiation emerges as an equilibrium response to capacity scarcity rather than only as a regulated/central design.

\paragraph{\textbf{Shmanske (1998, Contemporary Economic Policy).}}
Studies price discrimination in a setting with congestible service (e.g., usage of facilities where congestion affects quality), illustrating how pricing rules segment demand and allocate capacity when quality is endogenously degraded by congestion. The deep contribution is to operationalize the ``congestible service'' logic in a concrete market environment, reinforcing the interpretation of rationing/priorities as product differentiation.

\subsection{References (LaTeX \texttt{thebibliography} block)}\label{app:genealogy_refs}

\begin{thebibliography}{99}

\bibitem[Wilson(1989a)]{Wilson1989Ecta}
Wilson, R. (1989a).
Efficient and Competitive Rationing.
\textit{Econometrica}, 57(1), 1--40.

\bibitem[Wilson(1989b)]{Wilson1989JRE}
Wilson, R. (1989b).
Ramsey Pricing of Priority Service.
\textit{Journal of Regulatory Economics}, 1(3), 189--202.

\bibitem[Spulber(1992a)]{Spulber1992IER}
Spulber, D. (1992a).
Optimal Nonlinear Pricing and Contingent Contracts.
\textit{International Economic Review}, 33(4), 747--772.

\bibitem[Spulber(1992b)]{Spulber1992JRE}
Spulber, D. (1992b).
Capacity-Contingent Nonlinear Pricing by Regulated Firms.
\textit{Journal of Regulatory Economics}, 4(4), 299--319.

\bibitem[Woo(1990)]{Woo1990JRE}
Woo, C.-K. (1990).
Efficient Electricity Pricing with Self-Rationing.
\textit{Journal of Regulatory Economics}. % (fill vol/issue/pages if desired)

\bibitem[Strauss \& Oren(1993)]{StraussOren1993EJ}
Strauss, T. \& Oren, S. (1993).
Priority Pricing of Interruptible Electric Service with an Early Notification Option.
\textit{The Energy Journal}, 14(2), 175--196.

\bibitem[Doucet et al.(1996)]{DoucetEtAl1996EE}
Doucet, J., Min, K., Roland, M., \& Strauss, T. (1996).
Electricity rationing through a two-stage mechanism.
\textit{Energy Economics}. % (fill vol/issue/pages if desired)

\bibitem[Seeto et al.(1997)]{SeetoWooHorowitz1997EE}
Seeto, D., Woo, C.-K., \& Horowitz, I. (1997).
Time-of-use rates vs.\ Hopkinson tariffs redux: An analysis of the choice of rate structures in a regulated electricity distribution company.
\textit{Energy Economics}, 19(2), 169--185.

\bibitem[Woo et al.(1998)]{WooHorowitzMartin1998JRE}
Woo, C.-K., Horowitz, I., \& Martin, J. (1998).
Reliability Differentiation of Electricity Transmission.
\textit{Journal of Regulatory Economics}, 13(3), 277--292.

\bibitem[Bernard \& Roland(2000)]{BernardRoland2000REE}
Bernard, J.-T. \& Roland, M. (2000).
Load management programs, cross-subsidies and transaction costs: the case of self-rationing.
\textit{Resource and Energy Economics}, 22(2), 161--188.

\bibitem[Woo et al.(2002)]{WooHoriiHorowitz2002MDE}
Woo, C.-K., Horii, B., \& Horowitz, I. (2002).
The Hopkinson tariff alternative to TOU rates in the Israel Electric Corporation.
\textit{Managerial and Decision Economics}, 23(1), 9--19.

\bibitem[Horowitz \& Woo(2006)]{HorowitzWoo2006Energy}
Horowitz, I. \& Woo, C.-K. (2006).
Designing Pareto-superior demand-response rate options.
\textit{Energy}, 31(6), 1040--1051.

\bibitem[Woo et al.(2008)]{WooEtAl2008AMI}
Woo, C.-K., Kollman, E., Orans, R., Price, S., \& Horii, B. (2008).
Now that California has AMI, what can the state do with it?
\textit{Energy Policy}, 36(4), 1366--1374.

\bibitem[Greening(2010)]{Greening2010Energy}
Greening, L. (2010).
Demand response resources: Who is responsible for implementation in a deregulated market?
\textit{Energy}, 35(4), 1518--1525.

\bibitem[Moore et al.(2010)]{MooreEtAl2010Energy}
Moore, J., Woo, C.-K., Horii, B., Price, S., \& Olson, A. (2010).
Estimating the option value of a non-firm electricity tariff.
\textit{Energy}, 35(4), 1609--1614.

\bibitem[Matsukawa(2006)]{Matsukawa2006MPRA}
Matsukawa, I. (2006).
Regulating a Monopoly Offering Priority Service.
MPRA Paper No.\ 991.

\bibitem[Matsukawa(2009)]{Matsukawa2009JRE}
Matsukawa, I. (2009).
Regulatory effects on the market penetration and capacity of reliability differentiated service.
\textit{Journal of Regulatory Economics}, 36(2), 199--217.

\bibitem[Schroyen \& Oyenuga(2011)]{SchroyenOyenuga2011JRE}
Schroyen, F. \& Oyenuga, A. (2011).
Optimal pricing and capacity choice for a public service under risk of interruption.
\textit{Journal of Regulatory Economics}, 39(3), 252--272.

\bibitem[Chao(2011)]{Chao2011Baseline}
Chao, H.-P. (2011).
Demand response in wholesale electricity markets: the choice of customer baseline.
\textit{Journal of Regulatory Economics}, 39(1), 68--88.

\bibitem[Chao(2012)]{Chao2012Subscription}
Chao, H.-P. (2012).
Competitive electricity markets with consumer subscription service in a smart grid.
\textit{Journal of Regulatory Economics}, 41(1), 155--180.

\bibitem[Crampes \& L\'eautier(2015)]{CrampesLeautier2015JRE}
Crampes, C. \& L\'eautier, T.-O. (2015).
Demand response in adjustment markets for electricity.
\textit{Journal of Regulatory Economics}, 48(2), 169--193.

\bibitem[Clastres \& Khalfallah(2015)]{ClastresKhalfallah2015ENECO}
Clastres, C. \& Khalfallah, H. (2015).
An analytical approach to activating demand elasticity with a demand response mechanism.
\textit{Energy Economics}, 52, 195--206.

\bibitem[Woo et al.(2014)]{WooEtAl2014AppEne}
Woo, C.-K., Sreedharan, P., Hargreaves, J., Kahrl, F., Wang, J., \& Horowitz, I. (2014).
A review of electricity product differentiation.
\textit{Applied Energy}, 114, 262--272.

\bibitem[Woo et al.(2019)]{WooEtAl2019NiceDesign}
Woo, C.-K., Milstein, I., Tishler, A., \& Zarnikau, J. (2019).
A wholesale electricity market design sans missing money and price manipulation.
\textit{Energy Policy}, 134. % (article number/pages if desired)

\bibitem[David et al.(2007a)]{DavidLeBretonMerillon2007Gas}
David, L., Le Breton, M., \& M\'erillon, O. (2007a).
Regulating the Natural Gas Transportation Industry: Optimal Pricing Policy of a Monopolist with Advance-Purchase and Spot Markets.
IDEI Working Paper No.\ 488.

\bibitem[David et al.(2007b)]{DavidLeBretonMerillon2007Capacity}
David, L., Le Breton, M., \& M\'erillon, O. (2007b).
Public Utility Pricing and Capacity Choice with Stochastic Demand.
IDEI Working Paper No.\ 489.

\bibitem[Gale \& Holmes(1993)]{GaleHolmes1993AER}
Gale, I. \& Holmes, T. (1993).
Advance-Purchase Discounts and Monopoly Allocation of Capacity.
\textit{American Economic Review}. % (fill vol/issue/pages if desired)

\bibitem[Dana(1998)]{Dana1998JPE}
Dana, J. (1998).
Advance-Purchase Discounts and Price Discrimination in Competitive Markets.
\textit{Journal of Political Economy}, 106(2), 395--422.

\bibitem[Dana(2001)]{Dana2001IER}
Dana, J. (2001).
Monopoly Price Dispersion under Demand Uncertainty.
\textit{International Economic Review}. % (fill vol/issue/pages if desired)

\bibitem[Miravete(2002)]{Miravete2002ReStud}
Miravete, E. (2002).
Estimating Demand for Local Telephone Service with Asymmetric Information and Optional Calling Plans.
\textit{Review of Economic Studies}, 69(4), 943--971.

\bibitem[Herweg \& Mierendorff(2013)]{HerwegMierendorff2013JEEA}
Herweg, F. \& Mierendorff, K. (2013).
Uncertain Demand, Consumer Loss Aversion, and Flat-Rate Tariffs.
\textit{Journal of the European Economic Association}. % (fill vol/issue/pages if desired)

\bibitem[Reitman(1991)]{Reitman1991JIE}
Reitman, D. (1991).
Endogenous Quality Differentiation in Congested Markets.
\textit{Journal of Industrial Economics}. % (fill vol/issue/pages if desired)

\bibitem[Shmanske(1998)]{Shmanske1998CEP}
Shmanske, S. (1998).
Price Discrimination At The Links.
\textit{Contemporary Economic Policy}, 16(3), 368--378.

\end{thebibliography}

\end{document}